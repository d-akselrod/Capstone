\documentclass{article}

\usepackage{tabularx}
\usepackage{booktabs}

\title{Reflection Report on \progname}

\author{\authname}

\date{}

%% Comments

\usepackage{color}

\newif\ifcomments\commentstrue %displays comments
%\newif\ifcomments\commentsfalse %so that comments do not display

\ifcomments
\newcommand{\authornote}[3]{\textcolor{#1}{[#3 ---#2]}}
\newcommand{\todo}[1]{\textcolor{red}{[TODO: #1]}}
\else
\newcommand{\authornote}[3]{}
\newcommand{\todo}[1]{}
\fi

\newcommand{\wss}[1]{\authornote{blue}{SS}{#1}} 
\newcommand{\plt}[1]{\authornote{magenta}{TPLT}{#1}} %For explanation of the template
\newcommand{\an}[1]{\authornote{cyan}{Author}{#1}}

%% Common Parts

\newcommand{\progname}{ProgName} % PUT YOUR PROGRAM NAME HERE
\newcommand{\authname}{Team \#, Team Name
\\ Student 1 name
\\ Student 2 name
\\ Student 3 name
\\ Student 4 name} % AUTHOR NAMES                  

\usepackage{hyperref}
    \hypersetup{colorlinks=true, linkcolor=blue, citecolor=blue, filecolor=blue,
                urlcolor=blue, unicode=false}
    \urlstyle{same}
                                


\begin{document}

\maketitle

\plt{Reflection is an important component of getting the full benefits from a
learning experience.  Besides the intrinsic benefits of reflection, this
document will be used to help the TAs grade how well your team responded to
feedback.  In addition, several CEAB (Canadian Engineering Accreditation Board)
Learning Outcomes (LOs) will be assessed based on your reflections.}

\section{Changes in Response to Feedback}

\plt{Summarize the changes made over the course of the project in response to
feedback from TAs, the instructor, teammates, other teams, the project
supervisor (if present), and from user testers.}

\plt{For those teams with an external supervisor, please highlight how the feedback 
from the supervisor shaped your project.  In particular, you should highlight the 
supervisor's response to your Rev 0 demonstration to them.}

\subsection{Software Requirements Specification (SRS)}
Post-Rev 0, significant refinements were made to the SRS document. Early feedback pointed towards the necessity for a more user-centric approach, which led us to pivot from an AI model to an evidence-based algorithm for workout generation. The SRS was updated to reflect this fundamental change, focusing on the accuracy and reliability of the workout data provided.

Incorporating insights from kinesiology experts, the SRS was restructured to emphasize simplicity and evidence-based practices over predictive machine learning, aligning the app's core functionalities more closely with scientifically-backed fitness principles. This shift in approach resulted in the enhancement of user profiles, allowing for more personalized and adaptable workout plans.

\subsection{Hazard Analysis}
The Hazard Analysis document underwent a comprehensive update to assess the risks associated with the new direction of our project. With the transition to evidence-based content, we meticulously reviewed potential hazards related to incorrect workout information, user data security, and the physical safety of our users.

We integrated robust validation mechanisms to safeguard against the dissemination of harmful fitness advice and implemented more stringent security protocols for user data. Our revised hazard analysis also stressed the importance of educating users on safe workout practices to mitigate the risk of injury, a concern that was repeatedly echoed by our project supervisor.

\subsection{Supervisor Feedback Integration}
Feedback from our supervisor was especially influential after the Rev 0 demo. Our supervisor emphasized the need for workouts tailored to individual capabilities and the potential risks associated with generic fitness plans. This led us to refine our algorithm to accommodate user feedback and physiological data more effectively.

Moreover, the supervisor's insights into the healthcare implications of resistance training propelled us to prioritize user health outcomes in our design considerations. This translated into a heightened focus on workout variety, recovery, and adaptability within the app.

In conclusion, the iterative feedback process not only optimized the SweatSmart app for better performance but also aligned it more closely with the end-users' health and fitness objectives.

\subsection{Design and Design Documentation}

\subsection{VnV Plan and Report}

\section{Design Iteration (LO11)}

This section details the journey of our design process from the initial concept to the final implementation of SweatSmart, highlighting key pivots and refinements along the way.

\subsection{Initial Concept and Vision}
Our project commenced with the goal of harnessing artificial intelligence to generate personalized workout plans. However, the first iteration of our design faced significant challenges, particularly in terms of data availability and the broad scope of our target audience.

\subsection{Incorporating Expert Guidance}
The turning point in our design process came with the involvement of our supervisors, Dr. Stuart Philips and PhD candidate Bradley Currier. Their insights into exercise science and the kinesiology landscape prompted a shift towards an evidence-based approach, moving away from a predictive AI model to one grounded in established fitness principles.

\subsection{Refining Our Target Audience}
Feedback from our supervisors led to a redefinition of our target audience. We transitioned our focus from seasoned gym-goers to individuals new to working out or those who had been disengaged from regular exercise. This pivot embraced the notion that simplicity and accessibility are key in fostering a consistent fitness routine.

\subsection{Feature Evolution}
Subsequent design iterations saw the introduction and enhancement of several key features:

\begin{itemize}
  \item \textbf{Workout Generation:} Developed to deliver simple, evidence-based workout plans tailored to user-inputted goals and available resources.
  \item \textbf{Workout History:} Enabled users to track their progress and ensure safe advancement in their fitness journey.
  \item \textbf{Live Workout Tracking:} Allowed for real-time logging of workout activities, enriching the user's interactive experience with the app.
  \item \textbf{FitBot:} Our fitness chatbot was designed to provide quick and accurate responses to user inquiries.
  \item \textbf{Social Networking:} A feature aimed at fostering motivation through community engagement and social accountability.
\end{itemize}

\subsection{Usability Enhancements}
User-centered design was paramount, leading us to conduct thorough usability testing with kin experts. The feedback obtained from these sessions was invaluable in identifying and implementing user interface improvements and additional features, which significantly enhanced the overall user experience.

\subsection{Looking Forward}
While our project timeline constrained the extent of our feature set, the feedback and testing phase highlighted several areas for potential expansion. Future iterations could include progress visualization, a rewards system, and more comprehensive home screen functionality to drive user engagement and retention.

\subsection{Conclusion}
The evolution of SweatSmart's design is a testament to the power of collaboration, user feedback, and agility in the development process. The final design stands as a user-friendly, evidence-based fitness app that simplifies the workout experience while remaining adaptable and scientifically informed.

% End of Design Iteration section

\section{Design Decisions (LO12)}

\plt{Reflect and justify your design decisions.  How did limitations,
 assumptions, and constraints influence your decisions?}

\section{Economic Considerations (LO23)}

\plt{Is there a market for your product? What would be involved in marketing your 
product? What is your estimate of the cost to produce a version that you could 
sell?  What would you charge for your product?  How many units would you have to 
sell to make money? If your product isn't something that would be sold, like an 
open source project, how would you go about attracting users?  How many potential 
users currently exist?}

\section{Reflection on Project Management (LO24)}

\plt{This question focuses on processes and tools used for project management.}

\subsection{How Does Your Project Management Compare to Your Development Plan}

\plt{Did you follow your Development plan, with respect to the team meeting plan, 
team communication plan, team member roles and workflow plan.  Did you use the 
technology you planned on using?}

\subsection{What Went Well?}

\plt{What went well for your project management in terms of processes and 
technology?}

\subsection{What Went Wrong?}

\plt{What went wrong in terms of processes and technology?}

\subsection{What Would you Do Differently Next Time?}

\plt{What will you do differently for your next project?}

\end{document}
